% Exempel på färdig-formaterad sång till VN:s
% sångbok 2010.

% Denna fil kan användas som sådan, bara verserna,
% namnen och annan rådata behöver bytas ur fälten.
% Tecknet "%" markerar en kommentar som helt och 
% hållet ignoreras av programmet som läser filen.

% Spara den färdiga filen som 
% 'SangnamnUtanMellanslagEllerSkander.tex'
% t.ex. blir "Vid En Källa" till 
% 'VidEnKalla.tex'
% Varje sång blir en egen fil.

\beginsong{Båtlåt}[ 	% Börja sången här
	by={Robert Broberg},	% Författare
	sr={},			% Melodi
	index={Det var en båt}]		% Alternativa
			% sångnamn
	
\beginverse*		% Börja vers
Det var en båt som sa till en annan;
va du va stilig. Vi borde borda varann,
gjorda för varann och köla lite grann,
som bara båtar kan.
Badda bam bam bam bam
Badda bam bam bam.
\endverse			% Sluta vers

\beginverse*		% Börja vers
Andra båten sa;
klart att jag vill va
med och kryssa.
Kyssa din stiliga för,
i en stillsam slör,
vi varann förför.
som bara båtar gör.
Badda bam bam bam bam
Badda bam bam bam.
\endverse			% Sluta vers

\beginverse*		% Börja vers
Sen när det blir lä
ja, då kan vi klä av oss seglen.
Ligga en stund vid en boj,
skepp o'hoj.
Gnida vår fernissa lite grann och fnissa,
kasta (t)ankar.
Bli lite vågade, ha lite skoj,
oj, oj, oj!
\endverse			% Sluta vers

\beginverse*		% Börja vers
Och hur vi sedan få
en och kanske två egna små jollar,
jollrande efter på släp
i ett navelrep,
e en hemlighet,
som bara båtar vet.
Badda bam bam bam bam
Badda bam bam bam.
\endverse			% Sluta vers

\beginverse*		% Börja vers
Vi kan lägga till i äktenskapets hamn,
vid en brygga,
bygga ett båthus som vi kunde ligga i,
och tjära ner varann'
som bara båtar kan.
Badda bam bam bam bam
Badda bam bam bam.
\endverse			% Sluta vers
\endsong			% Sluta sång
