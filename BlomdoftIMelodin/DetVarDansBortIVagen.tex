% Exempel på färdig-formaterad sång till VN:s
% sångbok 2010.

% Denna fil kan användas som sådan, bara verserna,
% namnen och annan rådata behöver bytas ur fälten.
% Tecknet "%" markerar en kommentar som helt och 
% hållet ignoreras av programmet som läser filen.

% Spara den färdiga filen som 
% 'SangnamnUtanMellanslagEllerSkander.tex'
% t.ex. blir "Vid En Källa" till 
% 'VidEnKalla.tex'
% Varje sång blir en egen fil.

\beginsong{Det var dans bort i vägen}[ 	% Börja sången här
	by={Gustaf Fröding},	% Författare
	sr={}]		% Melodi
			% Alternativa
			% sångnamn
	
\beginverse*		% Börja vers
Det var dans bort i vägen på lördagsnatten, över
nejden gick låten 
av spelet och skratten,
det var tjoh! det var hopp! det var hej!
Nils Utterman, token och spelmansfante,
han satt med sitt bälgspel vid landsvägskanten,
för dudeli! dudeli! dej! 
\endverse			% Sluta vers

\beginverse*		% Börja vers
Där var Bolla, den präktiga Takeneflickan,
hon är fager och fin, men har intet i fickan,
hon är gäcksam och skojsam och käck.
Där var Kersti, den trotsiga, vandrande, vilda, där
var Finnbacka-Britta och Kajsa och Tilda
och den snudiga Marja i Bäck.
\endverse			% Sluta vers

\beginverse*		% Börja vers
Där var Petter i Toppsta
och Gusten i Backen,
det är pojkar, som orka att kasta på klacken
och att vischa en flick i skyn. 
Där var Flaxman på Torpet
och Niklas i Svängen
och rekryten Pistol och Högvaltadrängen
och Kall-Johan i Skräddarebyn. 
\endverse			% Sluta vers

\beginverse*		% Börja vers
Och de hade som brinnande blånor i kroppen,
och som gräshoppor hoppade Rejlandshoppen,
och mot stenar av klackar det small.
Och rockskörten flaxade, förkläden slängde,
och flätorna flögo och kjolarna svängde,
och musiken den gnällde och gnall. 
\endverse			% Sluta vers

\beginverse*		% Börja vers
In i snåret av björkar
och alar och hassel
var det viskande snack, det var tissel och tassel
bland de skymmande skuggorna där,
det var ras, det var lek över stockar och stenar,
det var kutter och smek under lummiga grenar
-- vill du ha mig, så har du mig här! 
\endverse			% Sluta vers

\beginverse*		% Börja vers
Över bygden låg tindrande stjärnfager natten,
det låg glimtande sken över skvalpande vatten
i den lövskogsbekransade sjön,
det kom doft ifrån klövern på blommande vallar
och från kådiga kottar på granar och tallar,
som beskuggade kullarnes krön. 
\endverse			% Sluta vers

\beginverse*		% Börja vers
Och en räv stämde in i den lustiga låten,
och en uv skrek uhu! ifrån Brynbärsbråten,
och de märkte, de hörde det ej.
Men uhu! hördes ekot i Getberget skria,
och till svar på Nils Uttermans dudelidia!
kom det dudeli! dudeli dej!
\endverse			% Sluta vers
\endsong			% Sluta sång
