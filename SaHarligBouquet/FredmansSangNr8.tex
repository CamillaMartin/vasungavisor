\beginsong{Fredmans sång nr 8}[
	by={Carl Michael Bellman},
	index={Ack, om vi hade}]
	
\beginverse*						
Ack, om vi hade, god' vänner, en så
ungerskt vin för vår strupa
och att vid såstången vi voro två,
som hade lov till att supa!
Vi skulle ligga — Gud signe Guds lån —
jag och kamraten med tungan i sån.
Ack, om vi hade, go vänner, en så
ungerskt vin för vår strupa!
\endverse		

\beginverse*						
Ja, fast den sån vore tunger som bly,
skull han bli lätt till att bära:
jag skull med krafter och rosende hy
lyfta min börda och svära,
och min kamrat skulle ta mig i famn,
dansa med stången och sjunga mitt namn.
Ack, om vi hade, go vänner, en så
ungerskt vin för vår strupa!
\endverse	

\beginverse*						
Ej på trehundrade steg någon själ,
nej, ingen käft på trehundra!
Nej, våra portar skull stängas så väl:
ingen skull bulta och dundra.
Bulta nån dit, skull vi lyfta vår stång,
fäkta som bröder och dö på en gång.
Ack, om vi hade, go vänner, en så
ungerskt vin för vår strupa!
\endverse	

\beginverse*						
Båd uti urväder, solsken och slask,
dunder och blixt och i torka,
ja, om från himlen föll änglar pladask
neder i sån, skull vi orka,
orka att slåss, köra opp dem till mån,
taga basunen och dricka ur sån.
Ack, om vi hade, go vänner, en så
ungerskt vin för vår strupa!
\endverse					
\endsong		
