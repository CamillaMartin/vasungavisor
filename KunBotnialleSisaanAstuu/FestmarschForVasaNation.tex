% Innehållet i Vasungavisor 2010

\beginsong{Festmarsch för Vasa nation}[
	by={Lars Huldén},
	sr={Fritiof Anderssons Paradmarsch},
	index={Från slätterna och havet}] 

\beginverse*
Från slätterna och havet har vi flyttat söderut.
Det gick för att det måste gå.
Nej, om vi här kysser marken!
Vi skrubbar oss i hjärnorna med lärdomens lut.
Men om vi vill
flaxar vi till
och flyger opp åt Kvarken.
Och inte ids vi buga oss för allt som kallas fint.
Vi klampar i salongerna, men klampar genuint.
Vi är en hop som häckar här
i söder några år.
Lever på lån
långt hemifrån
tills graden blivit vår.
\endverse

\beginverse*
Från slätten har vi med oss ner ett eget ögonmått.
Det siktar vi på världen med,
inte blir man imponerad!
Det finns så mycket stort som när det mäts blir ganska
smått.
Hej, om du vill
klämmer vi till
så du blir rent chockerad.
Men inte skall ni buga er som står här runt omkring
för också en som inte är en vasung är nånting.
Här kommer vi långt norrifrån
och är såna vi är.
Men om vi ser
noga på er
så är ni folk ni med.
\endverse

\beginverse*
Och sen så går det kanske som det ofta brukar gå:
Vi stannar här till livets slut.
Men inte glömmer vi marken
och stränderna och skogen där som barn vi härat på.
Nej, om vi vill
så slår vi till
och flyttar opp åt Kvarken.
Och nog för att vi inser att vi inte skiljer oss
så mycket ifrån andra, men det är en sak förstås:
Vi har ett hav, vi har en slätt
som stannat i vår blick.
Stannat hos dig,
stannat hos mig
vart vi i världen gick.
\endverse
\endsong
